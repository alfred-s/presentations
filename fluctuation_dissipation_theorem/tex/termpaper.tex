%%%%%%%%%%%%%%%%%%%%%%%%%%%%%%%%%%%%%%%%%%%%%%%%%%%%%%%%%%%%
%%  Presentation of Bachelor thesis. 
%%  A network model of the neocortex accounting for its laminar structure
%%  Friedrich Schuessler
%%  September 2015

\documentclass[xcolor=x11names,compress]{beamer}
\usepackage[utf8]{inputenc} 
\usepackage[T1]{fontenc}
\usepackage{cmbright}

%% General document %%%%%%%%%%%%%%%%%%%%%%%%%%%%%%%%%%
\usepackage{graphicx}
\usepackage{caption}  
\usepackage{subcaption}     % subfloats
\usepackage{amsmath,mathtools}  % math stuff

\usepackage{tikz}

\usepackage[style=authoryear]{biblatex}
\usetikzlibrary{decorations.fractals}
%%%%%%%%%%%%%%%%%%%%%%%%%%%%%%%%%%%%%%%%%%%%%%%%%%%%%%


%% Beamer Layout %%%%%%%%%%%%%%%%%%%%%%%%%%%%%%%%%%
\useoutertheme[subsection=false,shadow]{miniframes}
\setbeamerfont{title like}{shape=\scshape}
\setbeamerfont{frametitle}{shape=\scshape}

\setbeamercolor*{lower separation line head}{bg=DeepSkyBlue4} 
\setbeamercolor*{normal text}{fg=black,bg=white} 
\setbeamercolor*{alerted text}{fg=red} 
\setbeamercolor*{example text}{fg=black} 
\setbeamercolor*{structure}{fg=black} 

\usenavigationsymbolstemplate{}
\setbeamercolor*{palette tertiary}{fg=black,bg=black!10} 
\setbeamercolor*{palette quaternary}{fg=black,bg=black!10} 

\setbeamercovered{dynamic} % enable pause

\renewcommand{\(}{\begin{columns}}
\renewcommand{\)}{\end{columns}}
\newcommand{\<}[1]{\begin{column}{#1}}
\renewcommand{\>}{\end{column}}
%%%%%%%%%%%%%%%%%%%%%%%%%%%%%%%%%%%%%%%%%%%%%%%%%%
\useoutertheme{infolines} % authors, etc.

% Hide the "/ N" in slide "n / N":
\makeatletter 
\setbeamertemplate{footline}
{
  \leavevmode%
  \hbox{%
  %\begin{beamercolorbox}[wd=.333333\paperwidth,ht=2.25ex,dp=1ex,center]{author in head/foot}%
    %\usebeamerfont{author in head/foot}\insertshortauthor~~\beamer@ifempty{\insertshortinstitute}{}{(\insertshortinstitute)}
  %\end{beamercolorbox}%
  \begin{beamercolorbox}[wd=.666666\paperwidth,ht=2.25ex,dp=1ex,center]{title in head/foot}%
    \usebeamerfont{title in head/foot}\insertshorttitle
  \end{beamercolorbox}%
  \begin{beamercolorbox}[wd=.333333\paperwidth,ht=2.25ex,dp=1ex,right]{}%
    \usebeamerfont{date in head/foot}\hspace*{2em}
    \insertframenumber\hspace*{2ex} 
  \end{beamercolorbox}}%
  \vskip0pt%
}
\makeatother
%%%%%%%%%%%%%%%%%%%%%%%%%%%%%%%%%%%%%%%%%%%%%%%%%%

\bibliography{presentation.bib}

\DeclareGraphicsExtensions{.pdf,.png,.jpg, .eps}

%%%%%%%%%%%%%%%%%%%%%%%%%%%%%%%%%%%%%%%%%%%%%%%%%%%%%
% My own definitions
\usepackage{color, colortbl}  
\definecolor{TableColor}{HTML}{A1D99B}

% Math stuff
\newcommand*\diff{\mathop{}\!\text{d}}
\newcommand*\Diff[1]{\mathop{}\!\text{d^#1}}
\DeclareMathOperator{\Tr}{Tr} % Trace

%%%%%%%%%%%%%%%%%%%%%%%%%%%%%%%%%%%%%%%%%%%%%%%%%%%%%%%
\begin{document}

\begin{frame}{}
\title[FTD]{The Fluctuation-Dissipation Theorem}
\author{
Friedrich Schüßler}
\date{\normalsize 15 January 2016}
%\date{\today}
\titlepage

\centering 
Supervisor: Prof. Oliver Mülken \& Dr. Maxim Dolgushev 
\end{frame}

%\begin{frame}{Table of contents}
    %\tableofcontents
%\end{frame}

%\AtBeginSection[]{
%\begin{frame}{Table of contents}
    %\tableofcontents[currentsection]
%\end{frame}
%}

%%%%%%%%%%%%%%%%%%% Start
%%%%%%%%%%%%%%%%%%%%%%%%%%%%%%%%%%%%%%%%%%%%%%%%%%%%%%%%%%%%%%%%%%%%
\begin{frame}[t]{}
    \includegraphics[height=0.9\textheight]{../figures/timeline_full}
\end{frame}
%%%%%%%%%%%%%%%%%%%%%%%%%%%%%%%%%%%%%%%%%%%%%%%%%%%%%%%%%%%%%%%%%%%%
%%%%%%%%%%%%%%%%%%%%%%%%%%%%%%%%%%%%%%%%%%%%%%%%%%%%%%%%%%%%%%%%%%%%
\begin{frame}[t]{}
    \includegraphics[height=0.9\textheight]{../figures/timeline0}
\end{frame}
%%%%%%%%%%%%%%%%%%%%%%%%%%%%%%%%%%%%%%%%%%%%%%%%%%%%%%%%%%%%%%%%%%%%

\section{Historical Introduction: Brownian Motion}
\label{sec:brownian_motion}

% Brownian Motion
\begin{frame}{Brownian Motion}
    \includegraphics[height=0.8\textheight]{../figures/brownian}
\end{frame}

% Langevin
\begin{frame}[t]{Brownian Motion}
Langevin equation:
\begin{align*}
    \partial_t u &= -\gamma u + L(t) \\
    \text{with} \qquad \left \langle L(t) \right \rangle = 0; &\quad 
        \left \langle L(t)L(t') \right \rangle = \Gamma \delta(t - t') 
\end{align*}
\only<2-4>{
Average squared velocity:
\begin{equation*}
    \left \langle u^2(t)\right \rangle 
    = \frac{\Gamma}{2 \gamma} \left( 1 - e^{-2\gamma t} \right) 
\end{equation*}
}
\only<3-4>{
Equipartition law: 
\begin{equation*}
        \frac{1}{2} m \left \langle u^2 \right \rangle  
            = \frac{1}{2} k T
        %\quad \Rightarrow \quad \frac{\Gamma}{2\gamma} = \frac{kt}{m}
\end{equation*}
}
\only<4>{
\begin{equation*}
\implies \boxed{
            \Gamma = \frac{2\gamma kT}{m}
}
\end{equation*}
}
\end{frame}

 
% Perrin
\begin{frame}{Perrin's experiment (1908)}
\begin{equation*}
    \text{Displacement for $t \gg 1/\gamma$:} \qquad \qquad
\left \langle x^2(t) \right \rangle = \frac{2kT}{m\gamma} t
\end{equation*}
    \only<2>{
    \includegraphics[height=0.8\textheight]{../figures/perrin1}\\
    }
    \only<3>{
    \includegraphics[height=0.8\textheight]{../figures/perrin2}\\
}
    \only<4>{
    \includegraphics[height=0.8\textheight]{../figures/perrin3}\\
    }
    \only<5>{
    \includegraphics[height=0.8\textheight]{../figures/perrin4}\\
}
    %/His result: $7.15 \times 10^{23}\,\text{mol}^{-1}$ 
    %\quad (today: $6.022 \times 10^{23}\,\text{mol}^{-1}$)
\end{frame}

% Intro Pic
\begin{frame}{}
    \includegraphics[height=0.8\textheight]{../figures/intro}
\end{frame}

% Linear Response Theory

%%%%%%%%%%%%%%%%%%%%%%%%%%%%%%%%%%%%%%%%%%%%%%%%%%%%%%%%%%%%%%%%%%%%
\begin{frame}[t]{}
    \includegraphics[height=0.9\textheight]{../figures/timeline1}
\end{frame}
%%%%%%%%%%%%%%%%%%%%%%%%%%%%%%%%%%%%%%%%%%%%%%%%%%%%%%%%%%%%%%%%%%%%

\section{Linear Response Theory}
\label{sec:linear_response_theory}
\begin{frame}[t]{Response to a perturbation}
    \only<1>{
    \includegraphics[height=0.8\textheight]{../figures/linear_response0}
    }
    \only<2>{
    \includegraphics[height=0.8\textheight]{../figures/linear_response1}
    }
\end{frame}

\begin{frame}[t]{Perturbed system}
    Perturbation $f(t)$ coupling to operator $A$:
\begin{equation*}
    H(t) = H_0 - A \,f(t)
\end{equation*} 
\only<2->{%
Non-equilibrium average:
\begin{equation*}
\left \langle A(t) \right \rangle_{ne} = \frac{1}{Z}\Tr  \rho_0 A(t)
\end{equation*}
where 
\begin{align*}
    Z &= \text{partition function} \\
    \rho_0 &= \text{equilibrium density matrix}
\end{align*}
}
\end{frame}

\begin{frame}[t]{}
    Average response
\begin{align*}
\delta A(t) 
&:=  \left \langle A(t) \right \rangle_{ne} - \left \langle A \right \rangle_{eq} 
\only<2->{%
\intertext{
... can be linearized in $f(t)$:
}
    \delta A(t) 
    &= \int_{-\infty}^{\infty} \chi(t-t') f(t')\diff t'
    + \mathcal{O}(f^2) 
    }
    \only<3>{
\intertext{with response function}
\chi(t - t') &:= 
\frac{i}{\hbar}\,\theta(t - t')\,
\left \langle [A(t), A(t')] \right \rangle_{eq}
    }
    \only<4->{
\intertext{with response function}
\chi(t - t') &:= 
\frac{i}{\hbar} 
\underbrace{
\theta(t - t') 
}_\text{
causality
    }
\left \langle [A(t), A(t')] \right \rangle_{eq}
    }
\only<5>{
    \intertext{$A(t)$ evolves due to $H_0$:}
    A(t) &= e^{itH_0/\hbar}\,A\,e^{-itH_0/\hbar}
}
\end{align*}
\end{frame}

\begin{frame}[t]{Linear Response}
\begin{columns}[T] % contents are top vertically aligned
\begin{column}[T]{8.5cm} % each column can also be its own environment
    \begin{align*}
        \delta A(t) 
        &= \int_{-\infty}^{\infty} \chi(t-t') f(t')\diff t' \\
        \chi(t - t') &= 
        \frac{i}{\hbar} 
        \,\theta(t - t')\,
        \left \langle [A(t), A(t')] \right \rangle_{eq}
    \only<2->{
        \intertext{... a convolution.}}
    \only<3->{
        \intertext{Frequency space:}
        \Aboxed{
            \delta A(\omega) &= \chi(\omega) f(\omega) 
        }
    }
    \end{align*}
\end{column}
\begin{column}[T]{3cm} % alternative top-align that's better for graphics
    \only<4>{
    \includegraphics[height=0.8\textheight]{../figures/green_kubo}
    }
\end{column}
\end{columns}
\end{frame}

% Measure chi
\begin{frame}[t]{Measure $\chi(\omega)$}
\includegraphics[height=0.15\textheight]{../figures/measure_linear_response}
\only<2>{
\includegraphics[height=0.8\textheight]{../figures/measure_chi0}
}
\only<3>{
\includegraphics[height=0.8\textheight]{../figures/measure_chi1}
}
\only<4>{
\includegraphics[height=0.8\textheight]{../figures/measure_chi2}
}
\end{frame}

\section{Fluctuation-Dissipation Theorem}
\label{sec:fluctuation_dissipation_theorem}

%%%%%%%%%%%%%%%%%%%%%%%%%%%%%%%%%%%%%%%%%%%%%%%%%%%%%%%%%%%%%%%%%%%%
\begin{frame}[t]{}
    \includegraphics[height=0.9\textheight]{../figures/timeline2}
\end{frame}
%%%%%%%%%%%%%%%%%%%%%%%%%%%%%%%%%%%%%%%%%%%%%%%%%%%%%%%%%%%%%%%%%%%%

\begin{frame}[t]{The fluctuation-dissipation theorem}
... is based on thermal equilibrium properties!
\begin{itemize}
    \item Hamiltonian $H_0$
    \item 
        $\rho = \rho_0$  const., 
    \item 
        $A(t) = e^{iH_0 t/\hbar} A e^{-iH_0 t/\hbar}$.
    \item Observables: $\left \langle A(t) \right \rangle = \frac{1}{Z}\Tr  \rho_0 A(t)$
\end{itemize}
\vspace{0.2cm}
\only<2>{
\includegraphics[height=0.4\textheight]{../figures/fdt_scheme0}
}
\only<3>{
\includegraphics[height=0.4\textheight]{../figures/fdt_scheme1}
}
\end{frame}

\begin{frame}[t]{}
%\begin{equation*}
    %S(t-t') := \left \langle A(t)A(t') \right \rangle
%\end{equation*}
%\\
%\vspace{0.5cm}
Fluctuation -- autocorrelation function:
\begin{align*}
    S(t-t') &:= \left \langle A(t)A(t') \right \rangle
\only<2->{
\intertext{Dissipation -- response function:}
    \chi(\omega) &= \chi'(\omega) + i \chi''(\omega) 
}
\only<3->{
\intertext{In time domain:}
    \chi''(t - t')
    &= \frac{1}{2\hbar} \left \langle [A(t), A(t')] \right \rangle 
}
\only<4->{%
\intertext{Symmetries of both functions $\implies$}
    \Aboxed{
    \chi''(\omega) 
    &= \frac{1}{\hbar} \tanh \left( \frac{\hbar \omega}{2 k T} \right) S(\omega)
}
}
\end{align*}
\end{frame}

% Classical limit
\begin{frame}[t]{The classical limit, $\hbar \omega / kT \rightarrow 0$:}
\begin{equation*}
    \boxed{
        \chi''(\omega) 
            = \frac{\omega}{2 kT} S(\omega) 
    }
\end{equation*}
\\ \vspace{0.5cm}
\only<2>{
The protagonists next to Green and Kubo:
    \includegraphics[height=0.5\textheight]{../figures/nyquist}
}
\end{frame}


% Remarks
\begin{frame}[t]{What is experimentally accessible?}
\begin{align*}
    \quad \delta A  \qquad ~\text{or}~~& \qquad S \\
    \quad \text{response}  ~\qquad \text{or}~~& \qquad \text{autocorrelation}
\only<2-3>{
    \intertext{Connection:}
    \delta A \iff \chi \iff& \chi'' \iff S
}
\end{align*}
\only<3>{%
Kramers--Kronig relations:
\begin{align*}
    \chi 
        &= \chi' + i \chi'' \\
    \chi' 
        &= P \int\frac{\chi''(\omega')}{\omega'-\omega}\frac{\diff \omega'}{\pi} \\
    \chi''
        &= - P \int\frac{\chi'(\omega')}{\omega'-\omega}\frac{\diff \omega'}{\pi}
\end{align*}
}
\end{frame}

\begin{frame}[t]{Why "dissipation"?}
Displacement $A(t) = x(t)$, force $f(t) = f_0 \cos(\omega_0 t)$:
\only<2-4>{
\begin{align*}
    \delta x(t) 
    &= (x * f) (t) \\
    &= \chi'(\omega_0)f_0 \cos(\omega_0 t)
        + \chi''(\omega_0)f_0 \sin(\omega_0 t)
\end{align*}
}
\only<3-4>{
Dissipation $P$ = work per period $T$:
\begin{align*}
    P
        &= \frac{1}{T} \int_{0}^{T}f(t) \dot{\delta x}(t)\diff t  \\
        &= \frac{f_0}{2} \omega \chi''(\omega_0) 
\end{align*}
}
\only<4>{
In equilibrium: 
\begin{equation*}
S(\omega) > 0 
\quad \implies \quad 
\chi''(\omega) > 0
\quad \implies \quad 
    P > 0
\end{equation*}
}
\end{frame}



\section{Application: Microrheology}
\label{sec:application_microrheology}
%%%%%%%%%%%%%%%%%%%%%%%%%%%%%%%%%%%%%%%%%%%%%%%%%%%%%%%%%%%%%%%%%%%%
\begin{frame}[t]{}
    \includegraphics[height=0.9\textheight]{../figures/timeline3}
\end{frame}
%%%%%%%%%%%%%%%%%%%%%%%%%%%%%%%%%%%%%%%%%%%%%%%%%%%%%%%%%%%%%%%%%%%%

\begin{frame}[t]{Viscoelastic fluids}
\includegraphics[height=0.8\textheight]{../figures/viscoelastic}
\end{frame}

\begin{frame}[t]{Viscoelastic fluids}
\begin{columns}[T] % contents are top vertically aligned
\begin{column}[T]{6.5cm} % each column can also be its own environment
    Maxwell model: \\
    damper \quad + \quad spring:
    \begin{align*}
        \only<1-3>{
        \sigma &= \sigma_D = \sigma_S \quad \text{stress} \\
        \epsilon &= \epsilon_D + \epsilon_S \quad \text{strain} 
        }
        \only<2-3>{
    \intertext{Stress-strain relation}
        \eta \dot{\epsilon}(t) &= \sigma(t) + \tau \dot{\sigma}(t) \\\\
        }
        \only<3-3>{
        \implies \quad 
        \tilde{\sigma}(\omega) 
        &= - \frac{i \omega \eta}{1 - i \omega \tau} \tilde{\epsilon}(\omega)
        =: G^*(\omega) \tilde{\epsilon}(\omega)
        }
    \end{align*}    
    \only<3-3>{
    with complex shear modulus $G^*(\omega)$. 
    }
\end{column}
\begin{column}[T]{3cm} % alternative top-align that's better for graphics
        \includegraphics[height=0.2\textheight]{../figures/maxwell}
\end{column}
\end{columns}
\end{frame}

\begin{frame}[t]{Experiment: Bead in viscoelastic fluid}
\begin{columns}[T] % contents are top vertically aligned
\begin{column}[T]{7cm} % each column can also be its own environment
\only<2->{
Generalized Stokes-Einstein equation:
}
\begin{align*}
\only<2->{
    \tilde{f}(\omega) &= 6 \pi a \eta^*(\omega) \tilde{\dot{\epsilon}}(\omega) 
\intertext{with complex viscosity}
}
\only<2>{
    \eta^*(\omega)  &:= \frac{\eta}{1 - i\omega \tau}
\phantom{= \frac{G^*(\omega)}{-i\omega}}%
}
\only<3->{
    \eta^*(\omega)  &:= \frac{\eta}{1 - i\omega \tau}
    = \frac{G^*(\omega)}{-i\omega}%
}
%\intertext{For long time / low frequencies:}
%\eta^*(\omega) &\rightarrow \eta
\end{align*}
\only<4>{%
Linear response:
    \begin{align*}
    \phantom{
        \implies \quad 
    }
        \tilde{\epsilon}(\omega) 
        &= \chi(\omega) \tilde{f}(\omega) 
        = \frac{\tilde{f}(\omega)}{6\pi a G^*(\omega)} \\
    \end{align*}
}
\only<5->{%
Linear response:
    \begin{align*}
    \phantom{
        \implies \quad 
    }
        \tilde{\epsilon}(\omega) 
        &= \chi(\omega) \tilde{f}(\omega) 
        = \frac{\tilde{f}(\omega)}{6\pi a G^*(\omega)} \\
        \implies \quad 
        G^*(\omega) &= \frac{1}{6\pi a \chi(\omega)} 
    \end{align*}
}
\end{column}
\begin{column}[T]{3cm} % alternative top-align that's better for graphics
    \includegraphics[height=0.7\textheight]{../figures/particle_tracking}
\end{column}
\end{columns}
\end{frame}

\begin{frame}[t]{Experimental measurement of $G^*(\omega)$}
    Mason et al., 1997 
    \only<1-2>{
    \includegraphics[height=0.6\textheight]{../figures/complex_modulus1}
    }
    \only<3>{
    \includegraphics[height=0.6\textheight]{../figures/complex_modulus2}
    }
    \only<2->{
    \begin{equation*}
r(t) 
\implies 
S(\omega) 
\implies 
\chi''(\omega) 
\implies 
\chi(\omega)
\implies 
G^*(\omega)
    \end{equation*}
        }
\end{frame}

\section{Application: Active Hair-bundles}
\label{sec:application_active_hearing}
%%%%%%%%%%%%%%%%%%%%%%%%%%%%%%%%%%%%%%%%%%%%%%%%%%%%%%%%%%%%%%%%%%%%
\begin{frame}[t]{}
    \includegraphics[height=0.9\textheight]{../figures/timeline4}
\end{frame}
%%%%%%%%%%%%%%%%%%%%%%%%%%%%%%%%%%%%%%%%%%%%%%%%%%%%%%%%%%%%%%%%%%%%


\begin{frame}[t]{Is our system in equilibrium?}
Fluctuation-dissipation theorem:
\begin{equation*}
\chi''(\omega) 
    = \frac{\omega}{2 kT} S(\omega) 
\end{equation*}

Effective temperature:
\begin{equation*}
T_{\text{eff}}(\omega) := \frac{\omega}{2k}\frac{S(\omega)}{\chi''(\omega)}
\end{equation*}
\end{frame}


\begin{frame}[t]{Active hair-bundle motility}
    \only<1>{
    \includegraphics[height=0.8\textheight]{../figures/hairbundle1}
    }
    \only<2>{
    \includegraphics[height=0.8\textheight]{../figures/hairbundle2}
    }
    \only<3>{
    \includegraphics[height=0.8\textheight]{../figures/hairbundle3}
    }
\end{frame}


\begin{frame}[t]{Experimental setup and testing linear response}
    \only<1>{
    \includegraphics[height=0.8\textheight]{../figures/bundle_setup1}
    }
    \only<2>{
    \includegraphics[height=0.8\textheight]{../figures/bundle_setup2}
    }
    \only<3>{
    \includegraphics[height=0.8\textheight]{../figures/bundle_setup3}
    }
\end{frame}

\begin{frame}[t]{Experimental results}
    Martin et al., 2001
    \only<1>{
    \includegraphics[height=0.8\textheight]{../figures/bundle_results0}
    }
    \only<2>{
    \includegraphics[height=0.8\textheight]{../figures/bundle_results1}
    }
    \only<3>{
    \includegraphics[height=0.8\textheight]{../figures/bundle_results2}
}
    \only<4>{
    \includegraphics[height=0.8\textheight]{../figures/bundle_results3}
}
\end{frame}


% Conclusion
\section{Conclusion}
\label{sec:conclusion}
%%%%%%%%%%%%%%%%%%%%%%%%%%%%%%%%%%%%%%%%%%%%%%%%%%%%%%%%%%%%%%%%%%%%
\begin{frame}[t]{}
    \includegraphics[height=0.9\textheight]{../figures/timeline_full1}
\end{frame}
%%%%%%%%%%%%%%%%%%%%%%%%%%%%%%%%%%%%%%%%%%%%%%%%%%%%%%%%%%%%%%%%%%%%

\begin{frame}[t]{}
\large{Conclusion}
\normalsize
\begin{align*}
\text{perturbed system} 
~~\iff
~~~~\text{response func}&\text{tion}~~~~
\iff~~
\text{autocorrelation} 
&\\
\delta A 
\qquad \underset{1.}{\iff} \qquad 
\chi 
~~\underset{2.}{\iff}~~& \chi'' 
~~\quad \underset{3.}{\iff} \qquad S 
\end{align*}
\only<2->{
\begin{enumerate}
    \item Linear response theory
    \item Kramers-Kronig relations
    \item Fluctuation-dissipation theorem
\end{enumerate}
}
\only<3>{
\begin{itemize}
    \item Microrheology: 
        \begin{equation*}
        r(t) \implies S(\omega) \implies 
        \chi''(\omega) \implies \chi(\omega)
        \implies G^*(\omega)   
        \end{equation*}
\end{itemize}
}\only<4->{
\begin{itemize}
    \item Microrheology: 
        \begin{equation*}
        r(t) \implies S(\omega) \implies 
        \chi''(\omega) \implies \chi(\omega)
        \implies G^*(\omega)   
        \end{equation*}
    \item Active hair-bundles:
        \begin{equation*}
            S(\omega), \chi(\omega) 
            \implies T_\text{eff} \neq T \implies \text{actively driven}
        \end{equation*}
\end{itemize}
    }

\end{frame}


% References
\begin{frame}[t]
    \frametitle<presentation>{References}    

    \scriptsize
    Theory \& Historic review:\\
    \vspace{0.1cm}
    \tiny
    % Kubo
    Kubo, Rep. 
    The fluctuation-dissipation theorem.
    \textit{Reports on progress in physics} 29.1 (1966).
    \vspace{0.1cm}

    % Nonequilibrium
    Mazenko, Gene F.
    Nonequilibrium statistical mechanics. 
    \textit{John Wiley \& Sons}, 2008.
    \vspace{0.1cm}

    % Review
    Marconi, Umberto Marini Bettolo, et al.
    Fluctuation–dissipation: response theory in statistical physics.
    \textit{Physics reports} 461.4 (2008).
    \vspace{0.1cm}

    %History
    Newburgh, Ronald, Joseph Peidle, and Wolfgang Rueckner. 
    Einstein, Perrin, and the reality of atoms: 1905 revisited.
    \textit{American journal of physics} 74.6 (2006).
    \vspace{0.1cm}

    \scriptsize
    Microrheology:\\
    \vspace{0.1cm}
    \tiny
    % Bead in viscoelastic fluid
    Mason, T. G., et al.
    Particle tracking microrheology of complex fluids.
    \textit{Physical Review Letters} 79.17 (1997).
    \vspace{0.1cm}

    Mason, T. G., \& Weitz, D. A. \\
    Optical measurements of frequency-dependent 
    linear viscoelastic moduli of complex fluids.  
    \textit{Physical review letters}, 74(7), 1250, 1995. 
    \vspace{0.1cm}

    \scriptsize
    Active hair-bundle motility:\\
    \vspace{0.1cm}
    \tiny
    % Experimental setup bundle
    Benser, Michael E., Robert E. Marquis, and A. J. Hudspeth. \\
    Rapid, active hair bundle movements in hair cells from the bullfrog’s sacculus.
    \textit{The Journal of Neuroscience} 16.18 (1996).
    \vspace{0.1cm}

    % Linear Response of hair cells
    Martin, P., and A. J. Hudspeth. 
    Compressive nonlinearity in the hair bundle's active response to mechanical stimulation.
    \textit{Proceedings of the National Academy of Sciences} 98.25 (2001).
    \vspace{0.1cm}

    % F-D-T hair bundle
    Martin, P., A. J. Hudspeth, and F. Jülicher. 
    Comparison of a hair bundle's spontaneous oscillations with its response to 
    mechanical stimulation reveals the underlying active process. 
    \textit{Proceedings of the National Academy of Sciences} 98.25 (2001).
    \vspace{0.1cm}

    %% Further reading on hearing
    Reichenbach, Tobias, and A. J. Hudspeth. 
    The physics of hearing: fluid mechanics and the active process of the inner ear.
    \textit{Reports on Progress in Physics} 77.7 (2014).
    \vspace{0.1cm}

    % Cochlea
    %Hudspeth, A. J. 
    %Integrating the active process of hair cells with cochlear function.
    %\textit{Nature Reviews Neuroscience} 15.9 (2014).

\end{frame}


% Final
\begin{frame}[t]{}
    \includegraphics[height=0.8\textheight]{../figures/bullfrog}
\end{frame}




% Appendix
\section*{Appendix}
\label{sec:appendix}


\begin{frame}{}
    \huge Appendix
\end{frame}

% Backup: Heisenberg picture
\begin{frame}[t]{Heisenberg picture}
All time dependency carried by operators, while
$\rho = \rho_{eq}$ constant. \\
Operators:
\begin{equation*}
    A(t) = U^{-1}(t, t_0) A U(t, t_0)\,,
\end{equation*}
following the equations of motion
\begin{equation*}
    i \hbar \partial_t A(t) = [H(t), A(t)]\,.
\end{equation*}
Average measurements:
\begin{equation*}
    \left \langle A(t) \right \rangle_{ne} = \Tr  \rho_{eq} A(t)\,.
\end{equation*}
\end{frame}
    
\begin{frame}[t]{Deduction FDT}
Start with correlation function
\begin{equation*}
    \bar{S}_{AB}(t-t') 
        = \left \langle A(t)B(t') \right \rangle \,.
\end{equation*}

Using $A(t) = e^{itH/\hbar}A(t)e^{-itH/\hbar}$, one shows that
\begin{align*}
    \bar{S}_{AB}(t - t') 
&= \frac{1}{Z}\Tr e^{-\beta H} A(t)B(t') \\
&= \frac{1}{Z}\Tr e^{+i(t + i\beta\hbar) H/\hbar} Ae^{-i(t + i\beta\hbar) H / \hbar}
    e^{-\beta H}B(t') \\
&= \frac{1}{Z}\Tr A(t + i\beta\hbar)e^{-\beta H}B(t') \\
&= \frac{1}{Z}\Tr e^{-\beta H}B(t')A(t + i\beta\hbar) \\
&= \bar{S}_{BA}(t' - t - i\beta\hbar) \,.
\end{align*}
\end{frame}

\begin{frame}[t]{}
Then, with $\tau := t - t'$, 
\begin{align*}
\bar{S}_{AB}(\omega) 
&= \int e^{i\omega\tau} \bar{S}_{AB}(-\tau - i\beta\hbar)\diff \tau \\
&= \int e^{i\omega\tau} \left( \int
\bar{S}_{AB}(\omega') e^{-i\omega'(-\tau - i\beta\hbar)}
\frac{\diff \omega'}{2\pi}
\right)\diff \tau \\
&= \int e^{-\beta\hbar\omega'} 
\bar{S}_{AB}(\omega') 2\pi \delta(\omega + \omega')
\frac{\diff \omega'}{2\pi} \\
&= \bar{S}_{AB}(-\omega) e^{+\beta \hbar \omega}\,.
\end{align*}
And inserting yields
\begin{align*}
\chi_{AB}''(t - t')
    &= \frac{1}{2\hbar} \left \langle [A(t), B(t')] \right \rangle \\
    &= \frac{1}{2\hbar} \left \langle \bar{S}_{AB}(\tau) - \bar{S}_{BA}(-\tau) \right \rangle\,, \\
\Rightarrow \quad 
\chi_{AB}''(\omega) 
    &= \frac{1}{2\hbar}\left( \bar{S}_{AB}(\omega) - \bar{S}_{BA}(-\omega) \right)
    = \frac{1 - e^{-\beta \hbar \omega}}{2 \hbar} \bar{S}_{AB}(\omega) \,.
\end{align*}
\end{frame}

%\begin{frame}[t]
%\begin{equation*}
    %S_{AB}(t-t') 
        %= \left \langle [A(t),B(t')]_+ \right \rangle \quad ; \quad
    %\chi_{AB}''(t - t')
        %= \frac{1}{2\hbar} \left \langle [A(t), B(t')] \right \rangle\,.
%\end{equation*}
%\begin{align*}
    %S_{AB}(t-t') &= \left \langle [A(t),B(t')]_+ \right \rangle \\
    %\chi_{AB}''(t - t')
    %&= \frac{1}{2\hbar} \left \langle [A(t), B(t')] \right \rangle\,.
%\end{align*}
%\end{frame}


%\begin{frame}[t]{The FDT connects} 
%\vspace{1cm}
    %the symmetrized correlation function 
    %\begin{equation*}
        %S_{AB}(t-t') := \left \langle [A(t),B(t')]_+ \right \rangle
    %\end{equation*}
    %with the imaginary part of the response function
    %\begin{equation*}
    %\chi_{AB}''(t - t')
        %= \frac{1}{2\hbar} \left \langle [A(t), B(t')] \right \rangle\,.
    %\end{equation*}
%All calculations are done in equilibrium.
%\end{frame}

%% Deduction
%\begin{frame}[t]{Deduction}
    %is mainly based on symmetries: \\
    %Introducing
    %\begin{equation*}
        %\bar{S}_{AB}(t-t') := \left \langle A(t)B(t') \right \rangle\,,
    %\end{equation*}
    %one finds that
    %\begin{equation*}
        %\bar{S}_{AB}(\omega) = \bar{S}_{AB}(-\omega) e^{+\beta \hbar \omega}\,,
    %\end{equation*}
    %and thus
    %\begin{align*}
    %\chi_{AB}''(\omega) 
        %&= \frac{1}{2\hbar}\left( \bar{S}_{AB}(\omega) - \bar{S}_{BA}(-\omega) \right)\\
        %&= \frac{1 - e^{-\beta \hbar \omega}}{2 \hbar} \bar{S}_{AB}(\omega) \,.
    %\end{align*}
%\end{frame}


%\begin{frame}[t]{}
%For the symmetrized correlations, one gets
    %\begin{equation*}
    %\chi_{AB}''(\omega) 
    %= \frac{\omega}{2 E_\beta(\omega)} S_{AB}(\omega)
    %\end{equation*}
%where
%\begin{equation*}
    %E_\beta(\omega)
        %:= \frac{\omega \hbar}{2} \coth(\frac{\beta \hbar \omega}{2})\,.
%\end{equation*}\\
%In the classical limit $\hbar \omega / kT \rightarrow 0$, 
%\begin{equation*}
%\chi_{AB}''(\omega) 
    %= \frac{\omega}{2 kT} S_{AB}(\omega) \,.
%\end{equation*}
%\end{frame}


%\begin{frame}[t]{Schrödinger picture}
%System perturbed by field $h(t)$
%coupling to operator $B$: 
%\begin{equation*}
    %H(t) = H_0 - B h(t)
%\end{equation*} \\
%\vspace{1cm}
%\only<2-3>{
%-- Density matrix $\rho(t)$ time dependent \\
%-- Operators constant \\
%\vspace{1cm}
%}
%\only<3>{
%Macroscopic behavior: Averages of corresponding operators $A(t)$:
%\begin{equation*}
    %\left \langle A(t) \right \rangle_{ne} = \Tr  \rho(t) A
%\end{equation*}
%}
%\end{frame}

%% Kubo
%\begin{frame}[t]{}
%\includegraphics[height=0.8\textheight]{../figures/kubo}
%\end{frame}

%Interaction picture
%\begin{frame}[t]{Interaction picture}
%Time evolution operator $U(t, t')$ with
%\begin{equation*}
    %i \hbar \partial_t U(t, t') = H(t) U(t, t'); \quad U(t, t) = 1
%\end{equation*}
%Split: 
%\begin{align*}
    %\label{eq:}
    %U(t, t_0) &= U_0(t, t_0)\hat{U}(t, t_0) \\
    %\text{where} \quad U_0(t, t_0) &:= e^{iH_0(t-t_0) / \hbar} 
%\end{align*}
%\only<2-3>{
%Operators:
%\begin{equation*}
    %A^I(t) = U_0^{-1}(t, t_0) A U_0(t, t_0)
%\end{equation*}
%}
%\only<3>{
%Integral equation for $\hat{U}$:
    %\begin{align*}
        %\hat{U}(t, t_0) 
        %&= 1 - \frac{1}{i \hbar} \int_{t_0}^{t}B^I(t')h(t')\diff t'  + \mathcal{O}(h^2)
    %\end{align*}
%%Comparison yields
%%\begin{equation*}
    %%i\hbar \hat{U}(t, t_0)
        %%= - B^I(t) h(t) \hat{U}(t, t_0) 
%%\end{equation*}
%}
%\end{frame}
%\begin{frame}[t]{Linear approximation}
%Integral equation for $\hat{U}$:
    %\begin{align*}
        %\hat{U}(t, t_0) 
        %&= 1 - \frac{1}{i \hbar} \int_{t_0}^{t}B^I(t')h(t')\hat{U}(t', t_0)\diff t'  \\
        %&= 1 - \frac{1}{i \hbar} \int_{t_0}^{t}B^I(t')h(t')\diff t'  + \mathcal{O}(h^2)
    %\end{align*}
%\only<2>{
%Back to the average: 
%\begin{equation*}
%\left \langle A(t) \right \rangle_{ne} 
    %= \left \langle A \right \rangle_{eq} - 
    %\frac{1}{i \hbar} 
    %\int_{t_0}^{t} \left \langle [A^I(t), B^I(t)] \right \rangle_{eq} h(t')\diff t' 
    %+ \mathcal{O}(h^2)
%\end{equation*}
%}
%\end{frame}


%\begin{frame}[t]{Non-equilibrium average}
%\begin{align*}
%\left \langle A(t) \right \rangle_{ne} 
    %&= \Tr  \rho_{eq} A(t) \\
    %&= \left \langle A \right \rangle_{eq} - 
    %\frac{1}{i \hbar} 
    %\int_{t_0}^{t} \left \langle [A^I(t), B^I(t)] \right \rangle_{eq} h(t')\diff t' 
    %+ \mathcal{O}(h^2)
%\end{align*}
%%\begin{equation*}
%%\left \langle A(t) \right \rangle_{ne} 
    %%= \left \langle A \right \rangle_{eq} - 
    %%\frac{1}{i \hbar} 
    %%\int_{t_0}^{t} \left \langle [A^I(t), B^I(t)] \right \rangle_{eq} h(t')\diff t' 
    %%+ \mathcal{O}(h^2)
%%\end{equation*}
%\only<2-3>{%
%Average response:
%\begin{equation*}
%\delta A(t) 
%:=  \left \langle A(t) \right \rangle_{ne} - \left \langle A \right \rangle_{eq} 
%\end{equation*}
%}
%\only<3>{%
%Inserting yields 
%\begin{align*}
    %\delta A(t) 
    %&= 2i\int_{t_0}^{t} \chi_{AB}''(t, t') h(t')\diff t' \\ 
    %\intertext{where} 
    %\chi_{AB}''(t, t') 
        %&:= \frac{1}{2\hbar} \left \langle [A^I(t), B^I(t')] \right \rangle_{eq}
    %%&= \int_{-\infty}^{\infty}\chi_{AB}(t, t') h(t')\diff t' 
%\end{align*}
%}
%\end{frame}

%\begin{frame}[t]{Response function}
%\begin{columns}[T] % contents are top vertically aligned
%\begin{column}[T]{8.5cm} % each column can also be its own environment
    %Time-translation invariance
    %\begin{align*}
        %\chi''_{AB}(t, t') &= \chi''_{AB}(t - t') 
    %\only<2->{
    %\intertext{and linear response function}
        %\chi_{AB}(t - t')
        %&:= 2i \theta(t - t') \chi_{AB}''(t - t')
    %}
    %\only<3->{
    %\intertext{yield a convolution:}
        %\delta A(t) 
        %&= \int_{-\infty}^{\infty}\chi_{AB}(t - t') h(t')\diff t' 
    %}
    %\only<4->{
    %\intertext{or in frequency space:}
    %\Aboxed{
        %\delta A(\omega) &= \chi_{AB}(\omega) h(\omega) 
    %}
    %}
    %\end{align*}
%\end{column}
%\begin{column}[T]{3cm} % alternative top-align that's better for graphics
    %\only<5>{
    %\includegraphics[height=0.8\textheight]{../figures/green_kubo}
    %}
%\end{column}
%\end{columns}
%\end{frame}

%\begin{equation*}
    %\chi''_{AB}(t, t') = \chi''_{AB}(t - t')
%\end{equation*}
%...yields a convolution:
    %\delta A(t) 
    %&= \int_{-\infty}^{\infty}\chi_{AB}(t - t') h(t')\diff t' 



%\begin{align*}
    %\chi_{AB}(t - t')
    %&:= 2i \theta(t - t') \chi_{AB}''(t - t')
%\end{align*}
    
%\begin{frame}[t]{Finally}
%...with time-translation invariance we get:
%\begin{align*}
    %\chi_{AB}(t, t') &= \chi_{AB}(t - t') \\
    %\implies \quad 
    %\delta A(t) 
    %&= \int_{-\infty}^{\infty}\chi_{AB}(t - t') h(t')\diff t' 
%\end{align*}
%\only<2-3>{
%... a convolution! $\implies$
%%\begin{equation*}
    %%\delta A(t) = (\chi_{AB} * h)(t) 
    %%\quad \Leftrightarrow \quad  
        %%\delta A(\omega) = \chi_{AB}(\omega) h(\omega) 
%%\end{equation*}
%\begin{equation*}
    %\delta A(\omega) = \chi_{AB}(\omega) h(\omega) 
%\end{equation*}
%}
%\only<3>{
%The pioneers in the 1950s:
    %\includegraphics[height=0.4\textheight]{../figures/green}
%}
%\end{frame}





%%%%%%%%%%%%%%%%%%%%%%%% FDT
%\begin{equation*}
%\chi''(t - t')
    %= \frac{1}{2\hbar} \left \langle [A(t), A(t')] \right \rangle 
%\end{equation*}
%\only<3>{%
%Symmetries of both functions $\implies$
%\begin{equation*}
    %\boxed{
    %\chi_{AB}''(\omega) 
        %= \frac{\omega}{2 E_\beta(\omega)} S_{AB}(\omega)
    %}
%\end{equation*}
%average energy of harmonic oscillator
%\begin{equation*}
    %E_\beta(\omega)
        %:= \frac{\omega \hbar}{2} \coth\left(\frac{\beta \hbar \omega}{2} 
           %\right)  
%\end{equation*}\\
%}



%%%%%%%%%%%%%%%%%%%%%%%%%% Conclusion
%\begin{frame}[t]{Wrap-Up: 
%}
%\only<2-4>{
%Linear response:
%\begin{equation*}
    %\delta A(t) = (\chi_{AB} * h)(t) 
    %\quad \Leftrightarrow \quad  
        %\delta A(\omega) = \chi_{AB}(\omega) f(\omega) 
%\end{equation*}
%}\only<3-4>{
%Response function:
%\begin{align*}
    %\chi_{AB}(\omega) &= \chi'_{AB}(\omega) + i \chi''_{AB}(\omega)\\
    %\text{where} \quad 
    %\chi_{AB}''(t - t')
        %&= \frac{1}{2\hbar} \left \langle [A(t), B(t')] \right \rangle  \\
    %\chi_{AB}' 
        %&= P \int\frac{\chi_{AB}''(\omega')}{\omega'-\omega}\frac{\diff \omega'}{\pi} 
    %\quad \text{(Kramers--Kronig)}
%\end{align*}
%}
%\only<4>{
%The fluctuation-dissipation theorem:
%\begin{equation*}
    %\chi_{AB}''(\omega) = 
    %\begin{cases}
        %\frac{\omega}{2 E_\beta(\omega)} S_{AB}(\omega) 
        %& \text{(QM)} \\ \\
        %\frac{\omega}{2 kT} S_{AB}(\omega) 
        %& \text{(classical)}
    %\end{cases}
%\end{equation*}
%}
%\end{frame}

%\begin{frame}[t]{Conclusion}
%Relations between
%\begin{equation*}
%\delta A \iff \chi_{AB} \iff \chi''_{AB} \iff S_{AB}
%\end{equation*}
%\only<2-4>{
%Linear response theory:
%\begin{equation*}
    %\delta A(t) = (\chi_{AB} * h)(t) 
    %\quad \Leftrightarrow \quad  
        %\delta A(\omega) = \chi_{AB}(\omega) f(\omega) 
%\end{equation*}
%}\only<3-4>{
%The fluctuation-dissipation theorem:
%\begin{equation*}
    %\chi_{AB}''(\omega) = 
    %\begin{cases}
        %\frac{\omega}{2 E_\beta(\omega)} S_{AB}(\omega) 
        %& \text{(QM)} \\ \\
        %\frac{\omega}{2 kT} S_{AB}(\omega) 
        %& \text{(classical)}
    %\end{cases}
%\end{equation*}
%}
%\only<4>{
%Application:
%\begin{itemize}
    %\item Microrheology
%\item Testing equilibrium -- \emph{"Is a system actively driven?"}
%\end{itemize}
%}
%\end{frame}
\end{document}
