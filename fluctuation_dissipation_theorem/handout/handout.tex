%%%%%%%%%%%%%%%%%%%%%%%%%%%%%%%%%%%%%%%%%%%%%%%%%%%%%%%%%%%%%%%%%%%%%%
% How to use writeLaTeX: 
%
% You edit the source code here on the left, and the preview on the
% right shows you the result within a few seconds.
%
% Bookmark this page and share the URL with your co-authors. They can
% edit at the same time!
%
% You can upload figures, bibliographies, custom classes and
% styles using the files menu.
%
% If you're new to LaTeX, the wikibook is a great place to start:
% http://en.wikibooks.org/wiki/LaTeX
%
%%%%%%%%%%%%%%%%%%%%%%%%%%%%%%%%%%%%%%%%%%%%%%%%%%%%%%%%%%%%%%%%%%%%%%
\documentclass[a4paper]{tufte-handout}
 
%%%%%%%%%%%%%%%%%%%%%%%%%%%%%%%%%%%%%%%%%%%%%%%%%%%%%%%
% My own packages
\usepackage[utf8]{inputenc} 
\usepackage{booktabs}  % nice tables
\renewcommand{\arraystretch}{1.35}
\usepackage{subfigure}  % usually obsolete, but nothing else works
\renewcommand*{\thesubfigure}{}
%%%%%%%%%%%%%%%%%%%%%%%%%%%%%%%%%%%%%%%%%%%%%%%%%%%%%%%

%\geometry{showframe}% for debugging purposes -- displays the margins

\usepackage{amsmath}
\usepackage{array}  % table with math mode

% Set up the images/graphics package
\usepackage{graphicx}
\setkeys{Gin}{width=\linewidth,totalheight=\textheight,keepaspectratio}
\graphicspath{{graphics/}}

\title{The Fluctuation-Dissipation Theorem}
\author{Friedrich Schüßler}
\date{15 January 2016}  % if the \date{} command is left out, the current date will be used

% The following package makes prettier tables.  We're all about the bling!
\usepackage{booktabs}

% The units package provides nice, non-stacked fractions and better spacing
% for units.
\usepackage{units}

% The fancyvrb package lets us customize the formatting of verbatim
% environments.  We use a slightly smaller font.
\usepackage{fancyvrb}
\fvset{fontsize=\normalsize}

% Small sections of multiple columns
\usepackage{multicol}

% Provides paragraphs of dummy text
\usepackage{lipsum}


%%%%%%%%%%%%%%%%%%%%%%%%%%%%%%%%%%%%%%%%%%%%%%%%%%%%%%%

% These commands are used to pretty-print LaTeX commands
\newcommand{\doccmd}[1]{\texttt{\textbackslash#1}}% command name -- adds backslash automatically
\newcommand{\docopt}[1]{\ensuremath{\langle}\textrm{\textit{#1}}\ensuremath{\rangle}}% optional command argument
\newcommand{\docarg}[1]{\textrm{\textit{#1}}}% (required) command argument
\newenvironment{docspec}{\begin{quote}\noindent}{\end{quote}}% command specification environment
\newcommand{\docenv}[1]{\textsf{#1}}% environment name
\newcommand{\docpkg}[1]{\texttt{#1}}% package name
\newcommand{\doccls}[1]{\texttt{#1}}% document class name
\newcommand{\docclsopt}[1]{\texttt{#1}}% document class option name

%%%%%%%%%%%%%%%%%%%%%%%%%%%%%%%%%%%%%%%%%%%%%%%%%%%%%%%
% My own definitions
\usepackage{color, colortbl}  
\definecolor{TableColor}{HTML}{A1D99B}

%\makeatletter
%\renewcommand\fnum@figure{Fig. \thefigure}
%\makeatother

%\usepackage{caption}
\usepackage[font=footnotesize]{caption}
%\captionsetup[figure]{labelformat=simple, labelsep = period}
\makeatletter
\renewenvironment{marginfigure}[1][-1.2ex]%
  {\begin{@tufte@margin@float}[#1]{figure}%
    \captionsetup{labelformat=empty}}%, labelsep = period}}
  {\end{@tufte@margin@float}}
\makeatother


% Math stuff
\newcommand*\diff{\mathop{}\!\text{d}}
\newcommand*\Diff[1]{\mathop{}\!\text{d^#1}}
\DeclareMathOperator{\Tr}{Tr}

% Table with fixed width
\newcolumntype{L}[1]{>{\raggedright\let\newline\\\arraybackslash\hspace{0pt}}m{#1}}
\newcolumntype{C}[1]{>{\centering\let\newline\\\arraybackslash\hspace{0pt}}m{#1}}
\newcolumntype{R}[1]{>{\raggedleft\let\newline\\\arraybackslash\hspace{0pt}}m{#1}}

%%%%%%%%%%%%%%%%%%%%%%%%%%%%%%%%%%%%%%%%%%%%%%%%%%%%%%%
\begin{document}

\maketitle% this prints the handout title, author, and date

\begin{abstract}
\noindent The fluctuation-dissipation theorem (FDT) connects fluctuation properties
of a system in equilibrium with the response to a perturbation.
The proof is based on linear response theory.
Two possible applications are the 
measurement of the shear modulus of viscoelastic fluids,
as well as testing whether a system is actively driven.
\end{abstract}

\section{Historical introduction: Brownian motion}
\label{sec:historical_introduction}

\begin{marginfigure}[1cm]%
    \includegraphics[width=0.9\linewidth]{../figures/brownian_only}
\caption{Brownian motion: sample path}
\end{marginfigure}
\marginnote[0.3cm]{
Einstein's theory (1905) and Perrin's experiment (1908) confirmed the atomistic 
paradigm.
}

\begin{tabular}{L{5cm}  >{$}L{5cm}<{$}}
\toprule
Einstein-Smoluchowski & 
    D
        = \lim_{t \to \infty} \frac{ \left \langle x^2(t) \right \rangle}{2t} 
        = \frac{k T}{m \gamma}  \\
Langevin equation &
    \partial_t u = -\gamma u + L(t) \\
\qquad with &
    \left \langle L(t) \right \rangle = 0  \\
    &
        \left \langle L(t)L(t') \right \rangle = \Gamma \delta(t - t')  \\
    Equipartition law &
        \frac{1}{2} m \left \langle u^2 \right \rangle  
            = \frac{1}{2} k T  \\
FDT & 
    \left \langle L(t)L(t') \right \rangle = 2 \gamma^2 D  \delta(t - t') \\
\bottomrule
\end{tabular}


\section{Linear Response Theory}
\label{sec:linear_response_theory}

\begin{marginfigure}[0.5cm]%
\includegraphics[width=\linewidth]{../figures/linear_response1}
\caption{
Schematic response of observable $A(t)$ to perturbation $f(t)$.}
\setfloatalignment{b}
\end{marginfigure}

\marginnote[0.5cm]{
*~ Heisenberg picture (const. states $\rho$);\\  
$Z$ = partition function
}
\marginnote{
**~Operators evolve according to equilibrium Hamiltonian ($h(t) = 0$).
}
%\begin{tabular}{l  >{$}l<{$}}
\begin{tabular}{L{5cm}  >{$}L{5cm}<{$}}
\toprule
Hamiltonian & 
    H(t) = H_0 - B h(t) \\
Non-equilibrium average* &
    \left \langle A(t) \right \rangle_{ne}
    = \frac{1}{Z} \Tr \rho A(t) \\
Response &
\delta A(t) 
    :=  \left \langle A(t) \right \rangle_{ne} - \left \langle A \right \rangle_{eq} \\
    Response function**
    &
    \chi_{AB}(t,t')
    := \frac{i}{\hbar}\theta(t - t') \left \langle [A(t), B(t')] \right \rangle_{eq}\\
Linear response &
    \delta A(t) = (\chi_{AB} * h)(t)  \\
\qquad or &
        \delta A(\omega) = \chi_{AB}(\omega) h(\omega) \\
\bottomrule
\end{tabular}


\section{Fluctuation-dissipation theorem}

\begin{marginfigure}[0.5cm]%
\includegraphics[width=\linewidth]{../figures/measure_chi_only}
\caption{
    The FDT allows to assess $\chi''$ over a large, continuous 
    range of frequencies, as opposed to mechanical measurements of
    the response.
}
\setfloatalignment{b}
\end{marginfigure}
\marginnote[0.3cm]{
Energy of harmonic oscillator: \\
$
    E_\beta(\omega)
       = \frac{\omega \hbar}{2} \coth\left(\frac{\beta \hbar \omega}{2} 
           \right)  $
}
\marginnote{
Effective temperature:\\
$
T_{\text{eff}}(\omega) := \frac{\omega}{2k}\frac{S_{AB}(\omega)}{\chi''_{AB}(\omega)}
$
}

%\begin{tabular}{l  >{$}l<{$}}
\begin{tabular}{L{5cm}  >{$}L{5cm}<{$}}
\toprule
%Equilibrium &
    %H_0,\, \rho_0 \quad \text{const.} \\
Observables &
    A(t) = e^{iH_0 t/\hbar} A e^{-iH_0 t/\hbar} \\
Averages &
        \left \langle A(t) \right \rangle = \frac{1}{Z}\Tr  \rho_0 A(t) \\
Correlation function (sym.)  & 
    S_{AB}(t-t') = \left \langle [A(t),B(t')]_+ \right \rangle \\
Response function &
\chi_{AB}''(t - t')
= \frac{1}{2\hbar} \left \langle [A(t), B(t')] \right \rangle \\
FDT (quantum mechanical) &
    \chi_{AB}''(\omega) 
        = \frac{\omega}{2 E_\beta(\omega)} S_{AB}(\omega) \\
%Energy of harmonic oscillator &
    %E_\beta(\omega)
       %= \frac{\omega \hbar}{2} \coth\left(\frac{\beta \hbar \omega}{2} 
           %\right)  \\
FDT (classical) &
    \chi_{AB}''(\omega) 
        = \frac{\omega}{2 kT} S_{AB}(\omega) \\

Kramers-Kronig relations &
    \chi_{AB} 
        = \chi_{AB}' + i \chi_{AB}'' \\
        &
        \chi_{AB}' 
        = P \int\frac{\chi_{AB}''(\omega')}{\omega'-\omega}\frac{\diff \omega'}{\pi} \\
        %&
        %\chi_{AB}''
        %= - P \int\frac{\chi_{AB}'(\omega')}{\omega'-\omega}\frac{\diff \omega'}{\pi} \\
%Dissipation &
%B(t) = A(t) = x(t) \\
%& 
%h(t) = f_0 \cos(\omega t) \\
%& 
%P = \frac{f_0}{2} \omega \chi''(\omega)  \\
\bottomrule
\end{tabular} 


\section{Application: Mircorheology}
\label{sec:application_1_mircorheology}

\begin{marginfigure}[0.2cm]%
    \includegraphics[width=0.7\linewidth]{../figures/maxwell}
    \caption{Maxwell model for viscoelastic fluid:\\
    damper (viscosity $\eta$)
    and spring (elastic modulus $E$) in series;\\
   \quad  $\sigma$ = stress, \\
    \quad $\epsilon$ = strain.} 
  \label{fig:brownian}
\end{marginfigure}


%\begin{tabular}{l  >{$}l<{$}}
\begin{tabular}{L{5cm}  >{$}L{5cm}<{$}}
\toprule
Maxwell model &
    \eta \dot{\epsilon}(t) = \sigma(t) + \tau \dot{\sigma}(t) \\
Stress-strain relation &
    \tilde{\sigma}(\omega) 
    = G^*(\omega) \tilde{\epsilon}(\omega) \\
Complex shear modulus &
    G^*(\omega) = - \frac{i \omega \eta}{1 - i \omega \tau} \tilde{\epsilon}(\omega) \\
Stokes-Einstein equation &
        \tilde{f}(\omega) = 6 \pi a G^*(\omega) \tilde{\epsilon}(\omega) \\ 
Linear response &
        G^*(\omega) = \frac{1}{6\pi a \chi(\omega)}  \\
\bottomrule
\end{tabular} 

\marginnote[4cm]{
 \qquad   $G^*(\omega) = G'(\omega) + iG''(\omega)$
}
\begin{figure*}[h]
%\centering 
\subfigure[
    Mean square displacement of bead in viscoelastic fluid. 
    Inset: Recorded 2d trajectory.
]{
\includegraphics[width=0.35\linewidth]{../figures/msd_complex_modulus}
}~
\subfigure[
    Complex shear modulus $G^*(\omega)$ of DNA using 
    bead tracking. Diamonds: mechanical measurements.
]{
\includegraphics[width=0.34\linewidth]{../figures/complex_modulus_only}
}
\end{figure*}


\section{Application: Active hair-bundles -- Is hearing an active process?}
\label{sec:application_2_active_hair_bundles}

\begin{figure*}[h]
%\centering 
\subfigure[Experimental setup: flexible glass fiber connected to tip of hair-bundle.
]{
\includegraphics[width=0.32\linewidth]{../figures/bead1}
}~
\subfigure[
    Power spectrum $S(\omega)$ 
    and response function $\chi''(\omega)$.
    ]{
\includegraphics[width=0.27\linewidth]{../figures/bead3} 
}~
\subfigure[%
    Effective temperature of hair-bundle with (upper) 
    and without (lower) spontaneous oscillations.
    ]{
\includegraphics[width=0.30\linewidth]{../figures/bead4}
}
\end{figure*}

\subsection{References}
\label{sec:refernces}
\begin{fullwidth}
    \scriptsize
    %History
    Newburgh, R., J. Peidle, and W. Rueckner. 
    Einstein, Perrin, and the reality of atoms: 1905 revisited.
    \textit{American journal of physics} 74.6 (2006).
    \vspace{0.1cm}

    % Kubo
    \noindent
    Kubo, R. 
    The fluctuation-dissipation theorem.
    \textit{Reports on progress in physics} 29.1 (1966).
    \vspace{0.1cm}

    % Bead in viscoelastic fluid
    \noindent
    Mason, T. G., et al.
    Particle tracking microrheology of complex fluids.
    \textit{Physical Review Letters} 79.17 (1997).
    \vspace{0.1cm}

    % F-D-T hair bundle
    \noindent
    Martin, P., A. J. Hudspeth, and F. Jülicher. 
    Comparison of a hair bundle's spontaneous oscillations with its response to 
    mechanical stimulation reveals the underlying active process. 
    \textit{Proceedings of the National Academy of Sciences} 98.25 (2001).
\end{fullwidth}
\end{document}
